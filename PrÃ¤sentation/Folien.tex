% Dieser Text ist urheberrechtlich geschützt
% Er stellt einen Auszug eines von mir erstellten Referates dar
% und darf nicht gewerblich genutzt werden
% die private bzw. Studiums bezogen Nutzung ist frei
% Mai 2005 
% Autor: Sascha Frank 
% Universität Freiburg 
% www.informatik.uni-freiburg.de/~frank/

\documentclass{beamer}

\usetheme[progressbar=frametitle]{metropolis}
\setbeamertemplate{frame numbering}[fraction]
\useoutertheme{metropolis}
\useinnertheme{metropolis}
\usefonttheme{metropolis}
\usecolortheme{spruce}
\setbeamercolor{background canvas}{bg=white}

\usepackage[german]{babel}
\usepackage{german}
\title{Aufbau und Funktionsweise eines Prozessors}
\author{Marco Vogel}
\institute{Hochschule Hof}
\date{\today}

\begin{document}

\begin{frame}
\titlepage
\end{frame}

\begin{frame}{Gliederung}
\begin{enumerate}
\item{Bin\"are Zahlendarstellung}
\item{Komponenten eines Prozessors}
\item{Befehlsverarbeitung}
\item{Logisim}
\item{Beispiel}
\end{enumerate}
\end{frame}

\begin{frame}{Bin\"are Zahlendarstellung}
Beispiel: 135d = $\sum\limits_{i=0}^{n-1} a_i * 10^i$
\end{frame}


\end{document}

