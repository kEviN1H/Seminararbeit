% Dieser Text ist urheberrechtlich geschützt
% Er stellt einen Auszug eines von mir erstellten Referates dar
% und darf nicht gewerblich genutzt werden
% die private bzw. Studiums bezogen Nutzung ist frei
% Mai 2005 
% Autor: Sascha Frank 
% Universität Freiburg 
% www.informatik.uni-freiburg.de/~frank/

\documentclass{beamer}


\usepackage[german]{babel}
\usepackage{german}


\begin{document}
\title{Aufbau und Funktionsweise eines Prozessors}   
\author{Marco Vogel} 
\date{\today} 

\frame{\titlepage} 

\frame{\frametitle{Inhaltsverzeichnis}\tableofcontents} 


\section{Einleitung} 
\frame{\frametitle{Titel} 
Die einzelnen Frames sollte einen Titel haben 
}
\subsection{Bin\"are Zahlendarstellung}
\frame{ 
Denn ohne Titel fehlt ihnen was
}


\section{Abschnitt Nr. 2} 
\subsection{Listen I}
\frame{\frametitle{Aufz\"ahlung}
\begin{itemize}
\item Einf\"uhrungskurs in \LaTeX  
\item Kurs 2  
\item Seminararbeiten und Pr\"asentationen mit \LaTeX 
\item Die Beamerclass 
\end{itemize} 
}

\frame{\frametitle{Aufz\"ahlung mit Pausen}
\begin{itemize}
\item  Einf\"uhrungskurs in \LaTeX \pause 
\item  Kurs 2 \pause 
\item  Seminararbeiten und Pr\"asentationen mit \LaTeX \pause 
\item  Die Beamerclass
\end{itemize} 
}

\subsection{Listen II}
\frame{\frametitle{Numerierte Liste}
\begin{enumerate}
\item  Einf\"uhrungskurs in \LaTeX 
\item  Kurs 2
\item  Seminararbeiten und Pr\"asentationen mit \LaTeX 
\item  Die Beamerclass
\end{enumerate}
}
\frame{\frametitle{Numerierte Liste mit Pausen}
\begin{enumerate}
\item  Einf\"uhrungskurs in \LaTeX \pause 
\item  Kurs 2 \pause 
\item  Seminararbeiten und Pr\"asentationen mit \LaTeX \pause 
\item  Die Beamerclass
\end{enumerate}
}

\section{Abschnitt Nr.3} 
\subsection{Tabellen}
\frame{\frametitle{Tabellen}
\begin{tabular}{|c|c|c|}
\hline
\textbf{Zeitpunkt} & \textbf{Kursleiter} & \textbf{Titel} \\
\hline
WS 04/05 & Sascha Frank &  Erste Schritte mit \LaTeX  \\
\hline
SS 05 & Sascha Frank & \LaTeX \ Kursreihe \\
\hline
\end{tabular}}


\frame{\frametitle{Tabellen mit Pause}
\begin{tabular}{c c c}
A & B & C \\ 
\pause 
1 & 2 & 3 \\  
\pause 
A & B & C \\ 
\end{tabular} }


\section{Abschnitt Nr. 4}
\subsection{Bl\"ocke}
\frame{\frametitle{Bl\"ocke}

\begin{block}{Blocktitel}
Blocktext 
\end{block}

\begin{exampleblock}{Blocktitel}
Blocktext 
\end{exampleblock}


\begin{alertblock}{Blocktitel}
Blocktext 
\end{alertblock}
}
\end{document}

